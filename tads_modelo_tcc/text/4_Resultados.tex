\chapter{Resultados e discussões}
\label{Resultados}

Baseada na ciclização enzimática de GPP, nossas simulações foram aptas a gerar todos os componentes dispostos na figura ~\ref{figIntroCarbocation}. Junto com esses componentes, as simulações produzem um vasto montante de outros componentes acompanhados por todas as ciclizações e intermediários dos quais eles originam, formando uma extensa rede química de potenciais produtos de terpenos. Toda rede química pode ser convencionalmente exportada para um documento PDF.  Também, o hipergrafo, que é em si a própria rede química, pode ser computacionalmente acessado e processado para futuras análises. Além disso, é possível armazenar a rede química usando um banco de dados em grafos, como descrito em ~\cite{Silva2018graph}. Apesar de as simulações serem customizadas, o código fonte escrito pode não ser trivial para pesquisas não computacionais. A interface Web criada atenua esta condição, fazendo a customização da simulação mais intuitiva e menos capciosa, sendo suficientes alguns “drag and drop” e cliques.   Essa interface, combinada com uma imagem do Docker com todo o ambiente preparado, está disponível no \href{https://github.com/waldeyr/2PathTerpenes}{GitHub}\footnote{https://github.com/waldeyr/2PathTerpenes}.

\section{Percepções sobre a rede química gerada}

A combinação das regras de transformação de grafos e o número de iterações permitem explorar um amplo espaço químico possível de monoterpenos, expondo todos os mecanismos de ciclização enzimática de seu precursor, o GPP. Notavelmente, neste espaço químico, há três grupos de monoterpenos: os que são encontrados na natureza e são conhecidos, os que são encontrados na natureza, mas permanecem desconhecidos; e os que são fisicamente possíveis, mas provavelmente não serão encontrados na natureza devido ao alto custo de energia para sua produção. 

Avanços científicos na tecnologia como espectometria de massas combinada com cromatografia gasosa tem possibilitado a alocação de mais e mais monoterpenos do primeiro grupo, isto é, dos que são conhecidos. Outras tecnologias como engenharia genética tem possibilitado experimentos de diversas hipóteses na diversidade e rendimento da produção de monoterpenos nas plantas. É possível alterar a viabilidade do perfil de monoterpenos das plantas, como demonstrado por  Krasnyanski et al. no trabalho pioneiro deles com hortelã-pimenta, seguido por muitos outros como Lucker et
al., alcançando a produção de monoterpenos introduzindo TerPS de limão em plantas de tabaco. Fatores como a avaliabilidade do GPP, compartimentação subcelular, crescimento de plantas e custos físicos são problemas relevantes a serem considerados para engenharia metabólica das plantas. Como muitos fatores além da sequência primária influenciando o perfil de monoterpenos de uma TerpS, manter metadados sobre vias químicas é relevante para possibilitar a identificação de cenários da biossíntese de monoterpenos sob certas condições. Descobrir as possibilidades da produção de monoterpenos de acordo com determinadas circunstâncias pode ser útil para preditar a função das TerpS e consequentemente para engenharia metabólica das plantas.  Isso poderia ajudar a responder importantes questões biológicas, a exemplo: seria possível prever quais alterações seriam feitas ao modificar o perfil de monoterpenos de uma enzima comparando-a a um grupo de TerpS que produz a mesma variedade de monoterpenos? 


Terpenóides têm muitas aplicações, tanto ecológicas quanto comerciais, que são evidenciadas pelas características como moléculas voláteis influenciando o comportamento de insetos. 
A engenharia de produção de terpenóides nas plantas tem o potencial de melhor produzir o manejo de pragas ou polinização aprimorada, por exemplo. 

\section{Comparação com trabalhos relacionados}

Há algumas iniciativas de predições relevantes de vias metabólicas computacionais que consideram transformações intermediárias em reações químicas como  as RetroRules, Biotransformer, Isegawa et al., and Tian et al. Apesar de fornecerem resultados comparáveis, essas abordagens diferem uma das outras por usar distintos métodos e estrutura de dados para representar as moléculas. RetroRules e Biotransformer usam descrições de reações químicas codificadas por SMARTS e SMIRKS. Isegawa et. al. usaram o método AFIR para prever computacionalmente o caminho para formação de terpenos. Tian et al. apresentaram uma abordagem computacional para gerar todos os possíveis carbocátions da sínteses de monoterpenos definindo e organizando o espaço químico produzido. Esse trabalho tem objetivos e resultados similares a RetroRules, Biotransformer, and Tian et al., mas sua metodologia é bastante diferenciada. Chow et al. usou a abordagem de Tian et al. [6] para caracterizar a sintase de sesquiterpeno de Streptomyces clavuligerus, que colabora com a ideia que as redes químicas podem ajudar as tarefas funcionais das TerpS.


%\section{O que devo escrever aqui?}%
%Bem, quando você iniciou seu trabalho, você tinha um problema claro a resolver.%
%Você pesquisou a respeito do seu problema e também a respeito de ferramentas, tecnologias e outros recursos que poderiam ajudá-lo a resolver seu problema. No método você estabelecu os passos usados para resolver o problema. Agora é hora de mostrar em detalhes que você alcançou os objetivos definidos na \nameref{Introducao}.%

%Oriente-se pelos Objetivos. Descreva em detalhes o seu sucesso!!%