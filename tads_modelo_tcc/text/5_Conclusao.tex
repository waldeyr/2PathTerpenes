\chapter{Conclusão}
\label{Conclusao}

O método proposto neste trabalho provê uma forma valiosa para explorar a diversidade de monoterpenos expondo todo o processo de sua produção em uma estrutura de dados computacionalmente tratável. Ademais, a opção de salvar a rede química em um banco de dados torna-se assim particularmente interessante se estes metadados são armazenados juntos, o que faz ser possível explorá-los através de consultas. O uso de banco de dados em grafos para fins biológicos, embora recente, já está bem estabilizado com casos de grande sucesso. Além dos resultados aqui reportados, este trabalho traz um esboço de um sistema capaz de modelar uma rede química computacional e evidência da literatura experimental como metadados para melhor guiar a tarefa funcional das Terps. Tal sistema é um trabalho futuro que substituirá o banco de dados 2Path.
